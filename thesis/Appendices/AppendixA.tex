% Appendix A

\chapter{Frequently Asked Questions} % Main appendix title

\label{AppendixA} % For referencing this appendix elsewhere, use \ref{AppendixA}

\section{How do I change the colors of links?}

The color of links can be changed to your liking using:

{\small\verb!\hypersetup{urlcolor=red}!}, or

{\small\verb!\hypersetup{citecolor=green}!}, or

{\small\verb!\hypersetup{allcolor=blue}!}.

\noindent If you want to completely hide the links, you can use:

{\small\verb!\hypersetup{allcolors=.}!}, or even better: 

{\small\verb!\hypersetup{hidelinks}!}.

\noindent If you want to have obvious links in the PDF but not the printed text, use:

{\small\verb!\hypersetup{colorlinks=false}!}.

\section{Experimenting with toy problems}

Comparar invariantes de Vapnik con los nuevos

Datasets

- Moons

- Cluster points

- Circles


Cosas que se podrían probar: Influencia de los hiperparámetros


\section{Experimentation on real data}
\subsection{Experimental settings}
{\bf Data:} Datos y tabla resumen de los datasets\\
{\bf Baselines:} Que métodos vas a comparar.
SVM sin invariantes
SVM con invariantes Vapnik
SVM con invariantes Vladis

Todos con el mismo sistema de selección.
\\
{\bf Methodology and performance metrics:} XValidation? cuanto, como mides los resultados?

Accuracy promedio i desviación estandar.
\\
{\bf Experimental and parameter setting:}
Haces parameter tunning?

Que experimentos propones?

1- Experimento base: Todos los datos con los modelos a comparar\\
2- Experimento con reduccion de datos.


\subsection{Results}
