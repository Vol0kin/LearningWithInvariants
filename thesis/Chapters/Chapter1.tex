% Chapter Template

\chapter{Introduction}

\label{Chapter1}

%----------------------------------------------------------------------------------------
%	SECTION 1: MOTIVATION AND BRIEF DESCRIPTION OF THE PROJECT
%----------------------------------------------------------------------------------------

\section{Motivation and brief description of the project}

The current machine learning paradigms consider no further information apart from the data
samples when trying to learn a model that can represent the data and be used later on
to infer information about new samples. During the training process, the current methods
try to minimize the error with respect to the original data, thus searching the function
that best fits the data in an infinite space of functions. However, the training data has
statistical information that can help reduce the searching space to a region of it, allowing
also to find functions that better fit the data.

A new learning paradigm that takes into account the statistical information of the training data
in the form of statistical invariants has been recently proposed by \cite{Vapnik2019}. Thanks to it,
statistical information of the problem can be used in the learning process, which might
be overlooked by most of the models because some relationships between variables are hard to
spot or require prior knowledge that the model does not have access to. Nonetheless,
it seems that this learning paradigm has not been fully explored or applied that much in practice.

Therefore, in this work we would like to further explore the possible applications of this
paradigm and whether it can be made more general, without requiring prior knowledge of the problem.

%----------------------------------------------------------------------------------------
%	SECTION 2: GOALS AND OBJECTIVES
%----------------------------------------------------------------------------------------

\section{Goals and objectives}

In this thesis we aim to
\begin{enumerate*}[label=(\roman*)]
    \item understand and further explore the application of the invariants in the learning
    problem,
    \item propose new invariants that require no previous knowledge of the problem and
    thus can be applied to multiple domains,
    \item automatize the selection process of the most suitable invariants for a new problem and
    \item extend the learning paradigm so that it can be applied to multiclass classification problems.
\end{enumerate*}

In order to accomplish these goals, we propose a series of milestones that must be achieved
first:

\begin{enumerate}
    \item Understand the original paper and reproduce it, which implies implementing the proposed
    algorithms for learning with statistical invariants and reproducing some of the experiments
    and results. Because there is no source code available, we have to start from scratch.
    \item Propose new invariants and apply them to the same problems as the ones in the paper
    to get an initial idea of how they work.
    \item Build a wrapper around the previously defined binary classifier to enable its use
    in multiclass classification problems.
    \item Experiment with multiclass classification problems to test the proposed invariants and
    compare them to a baseline to see whether they are actually helping or not during the learning
    process.
\end{enumerate}

The expected outcome of this work is a small software package that contains a machine learning
model that can be applied to classification problems.

%----------------------------------------------------------------------------------------
%	SECTION 3: BRIEF SUMMARY OF THE RESULTS
%----------------------------------------------------------------------------------------

\section{Brief summary of the results}

% TODO: Waiting for final results

%----------------------------------------------------------------------------------------
%	SECTION 4: LAYOUT
%----------------------------------------------------------------------------------------

\section{Layout}

This thesis is structured as follows:

\begin{itemize}
    \item Chapter \ref{Chapter1} introduces this work, presenting the main goals that are expected
    to be achieved by the end of it and briefly discussing the obtained results.
    \item Chapter \ref{Chapter2} explains the background work that has inspired this project,
    showing its main contributions and results.
    \item Chapter \ref{Chapter3} proposes a series of invariants that aim to be more general and easy
    to apply to different problems and a method to expand this paradigm to multiclass classification
    problems.
    \item Chapter \ref{Chapter4} studies how the proposed invariants and methods work in practice
    and what results can be achieved with them.
    \item Chapter \ref{Chapter5} briefly discusses what conclusions can be drawn from this work
    and what future work can be done on the topic.
\end{itemize}

