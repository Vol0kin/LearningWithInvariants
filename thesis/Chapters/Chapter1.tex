% Chapter Template

\chapter{Introduction}

\label{Chapter1}

%----------------------------------------------------------------------------------------
%	SECTION 1: MOTIVATION AND BRIEF DESCRIPTION OF THE PROJECT
%----------------------------------------------------------------------------------------

\section{Motivation and brief description of the project}

The current machine learning paradigms consider no further information apart from the data
samples when trying to learn a model that can represent the data and be used later on
to infer information about new samples. During the training process, the current methods
try to minimize the error with respect to the original data, thus searching the function
that best fits the data in an infinite space of functions. However, the training data has
statistical information that can help reduce the searching space to a region of it, allowing
also to find functions that better fit the data.

A new learning paradigm that takes into account the statistical information of the training data
in the form of statistical invariants has been recently proposed by \cite{Vapnik2019}. Thanks to it,
statistical information of the problem can be used in the learning process, which might
be overlooked by most of the models because some relationships between variables are hard to
spot or require prior knowledge that the model does not have access to. Nonetheless,
it seems that this learning paradigm has not been fully explored or applied that much in practice.
Therefore, in this work we would like to further explore the possible applications of this
paradigm and whether it can be made more general, without requiring prior knowledge of the problem.

%----------------------------------------------------------------------------------------
%	SECTION 2: GOALS AND OBJECTIVES
%----------------------------------------------------------------------------------------

\section{Goals and objectives}

In this thesis we aim to
\begin{enumerate*}[label=(\roman*)]
    \item understand and further explore the application of the invariants in the learning
    problem,
    \item propose new invariants that require no previous knowledge of the problem and
    thus can be applied to multiple domains and
    \item create a small module that implements these invariants and allows them
    to be applied to binary and multiclass classification problems.
\end{enumerate*}

In order to accomplish this work, there are some milestones that must be achieved:

\begin{enumerate}
    \item Understand and reproduce the methods and some of the experiments and results of the
    original paper. Because there is no source code available, we have to start from scratch.
    \item Propose new invariants and apply them to the same problems as the ones in the paper
    to get an initial idea of how they work.
    \item Build a wrapper around the binary classifier to enable its use in multiclass classification
    problems.
    \item Experiment with multiclass classification problems to test the proposed invariants and
    compare them to a baseline to see whether they are actually helping or not during the learning process.
\end{enumerate}


%----------------------------------------------------------------------------------------
%	SECTION 3: BRIEF SUMMARY OF THE RESULTS
%----------------------------------------------------------------------------------------

\section{Brief summary of the results}

% TODO: Waiting for final results

%----------------------------------------------------------------------------------------
%	SECTION 4: LAYOUT
%----------------------------------------------------------------------------------------

\section{Layout}

This thesis is structured as follows. In Chapter \ref{Chapter2} we are going to discuss the
background work that has inspired this project, showing its main contributions and results.
Then, in Chapter \ref{Chapter3} we are going to present a series of proposals that
aim to make the original work more general and easy to apply to different problem. Following
the description of the proposed methods, in Chapter \ref{Chapter4} we are going to study how
they work in practice and see what results can be achieved with them. Finally, in Chapter
\ref{Chapter5} we are going to briefly discuss what conclusions can be drawn from this work
and what the future work that can be done.

%----------------------------------------------------------------------------------------
%	SECTION 2
%----------------------------------------------------------------------------------------

