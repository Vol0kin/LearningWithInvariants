% Chapter Template

\chapter{Proposals} % Main chapter title
\label{Chapter3}

As we have already seen in Chapter \ref{Chapter2}, LUSI introduces a new data-driven learning
paradigm which aims to find better approximations of the conditional probability function using
statistical invariants. However, these invariants are often problem dependent and choosing the
appropriate ones is not an easy task, as there are many possible ones and some of them are not
straightforward to come up with.

Thus, our main goal with this work is to create a series of new invariants which we expect to be of
general use, as well as automatizing the selection process of the invariants that should be considered
for a given problem and extending the LUSI paradigm to multiclass problems\footnote{Even though in the
original paper the authors state that the method can be applied to multiclass problems, they do not go
into too much detail of how this is done.}.

In this chapter we are going to present our proposals, which are a two new invariants based on randomness
which aim to be more general than the original ones and an extension of the LUSI paradigm to multiclass problems
using Error Correcting Output Codes (ECOC).

\section{Random invariants}

Our first proposal is a series of random invariants, which are invariants that have some sort of random
process inside of them but that aim to preserve some sort of statistical information of the data. This way,
we aim to greatly reduce the amount of necessary prior knowledge of the problem when choosing which invariants
to use, making it just another hyperparameter of a machine learning model. In this work, we propose
an invariant based on random projections as well as another one based on random hyperplanes.

\subsection{Random projections}

\subsection{Random hyperplanes}

\section{Extending the LUSI paradigm to multiclass problems}

