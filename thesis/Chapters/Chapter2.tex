% Chapter Template

\chapter{Learning using statistical invariants} % Main chapter title
\label{Chapter2}

Given that this work intends to explore the applications of the invariants in the learning
process, we first need to introduce the background work that proposed this new learning
paradigm, which is called LUSI (Learning Using Statistical Invariants).

This chapter intends to provide the necessary background to understand the basis of this work
and an overview of the most relevant aspects of the original paper that presented the LUSI paradigm,
which was proposed by \cite{Vapnik2019}. For further information and more details, please
refer to the original paper.

%----------------------------------------------------------------------------------------
%	SECTION 1
%----------------------------------------------------------------------------------------
\section{Weak convergence and the LUSI paradigm}
\subsection{Weak vs strong}
\subsection{LUSI}
\subsection{Predicate selection}


\section{Statistical invariants}

An statistical invariant is a specific realization of a predicate with statistical meaning.



Before talking about the LUSI paradigm we need to define the concept of statistical invariants.
An \emph{invariant} is a mathematical property that remains unchanged after a transformation
or operation. Hence, a \emph{statistical invariant} is a statistical property that remains
unchanged after a transformation or operation.

There are many different statistical invariants, each one of them providing different information
about the data. Consider a binary classification problem in which we have elements from a
positive and a negative class labeled as 1 and 0, respectively. We could use some statistical
invariants of the training data in order to reduce the number of possible functions during
the training process. For instance, we could use the zeroth moment invariant, which tells the
proportion of elements in each class. This way, the model has to both correctly label each sample
and keep the proportion of elements of each class. Also, we could use a first moment invariant,
which would tell us the mean of each class.


\subsection{Zero-order invariants}

\subsection{First-order invariants}




\subsection{Solving the learning problem}

\section{Main results}

La seleccion del invariante es problem dependent y es un "black-art".
