%%%%%%%%%%%%%%%%%%%%%%%%%%%%%%%%%%%%%%%%%
% Masters Thesis 
% LaTeX Template
%
% This template is based on a template by:
% Steve Gunn (http://users.ecs.soton.ac.uk/srg/softwaretools/document/templates/)
% Sunil Patel (http://www.sunilpatel.co.uk/thesis-template/)
%
% Template license:
% CC BY-NC-SA 3.0 (http://creativecommons.org/licenses/by-nc-sa/3.0/)
%
%%%%%%%%%%%%%%%%%%%%%%%%%%%%%%%%%%%%%%%%%

%----------------------------------------------------------------------------------------
%	PACKAGES AND OTHER DOCUMENT CONFIGURATIONS
%----------------------------------------------------------------------------------------

\documentclass[
11pt, % The default document font size, options: 10pt, 11pt, 12pt
%oneside, % Two side (alternating margins) for binding by default, uncomment to switch to one side
english, % ngerman for German
singlespacing, % Single line spacing, alternatives: onehalfspacing or doublespacing
%draft, % Uncomment to enable draft mode (no pictures, no links, overfull hboxes indicated)
%nolistspacing, % If the document is onehalfspacing or doublespacing, uncomment this to set spacing in lists to single
%liststotoc, % Uncomment to add the list of figures/tables/etc to the table of contents
%toctotoc, % Uncomment to add the main table of contents to the table of contents
%parskip, % Uncomment to add space between paragraphs
%nohyperref, % Uncomment to not load the hyperref package
headsepline, % Uncomment to get a line under the header
%chapterinoneline, % Uncomment to place the chapter title next to the number on one line
%consistentlayout, % Uncomment to change the layout of the declaration, abstract and acknowledgements pages to match the default layout
]{MastersDoctoralThesis} % The class file specifying the document structure

\usepackage[utf8]{inputenc} % Required for inputting international characters
\usepackage[T1]{fontenc} % Output font encoding for international characters

\usepackage{mathpazo} % Use the Palatino font by default

\usepackage[backend=bibtex,style=authoryear,natbib=true]{biblatex} % Use the bibtex backend with the authoryear citation style (which resembles APA)

\addbibresource{example.bib} % The filename of the bibliography

\usepackage[autostyle=true]{csquotes} % Required to generate language-dependent quotes in the bibliography

% Extra packages
\usepackage[inline]{enumitem}
\usepackage{amsmath}
\usepackage{float}
\usepackage{amsmath}
\usepackage{tablefootnote}
\usepackage{algorithm}
\usepackage{algpseudocode}

\DeclareMathOperator*{\argmin}{\arg\!\min}
%----------------------------------------------------------------------------------------
%	MARGIN SETTINGS
%----------------------------------------------------------------------------------------

\geometry{
	paper=a4paper, % Change to letterpaper for US letter
	inner=2.5cm, % Inner margin
	outer=3.8cm, % Outer margin
	bindingoffset=.5cm, % Binding offset
	top=1.5cm, % Top margin
	bottom=1.5cm, % Bottom margin
	%showframe, % Uncomment to show how the type block is set on the page
}

%----------------------------------------------------------------------------------------
%	THESIS INFORMATION
%----------------------------------------------------------------------------------------

\thesistitle{Learning with invariants} % Your thesis title, this is used in the title and abstract, print it elsewhere with \ttitle
\supervisor{Dr. Oriol \textsc{Pujol}} % Your supervisor's name, this is used in the title page, print it elsewhere with \supname
\examiner{} % Your examiner's name, this is not currently used anywhere in the template, print it elsewhere with \examname
\degree{MSc} % Your degree name, this is used in the title page and abstract, print it elsewhere with \degreename
\author{Vladislav \textsc{Nikolov}} % Your name, this is used in the title page and abstract, print it elsewhere with \authorname
\addresses{} % Your address, this is not currently used anywhere in the template, print it elsewhere with \addressname

\subject{Computer Science} % Your subject area, this is not currently used anywhere in the template, print it elsewhere with \subjectname
\keywords{} % Keywords for your thesis, this is not currently used anywhere in the template, print it elsewhere with \keywordnames
\university{\href{http://www.ub.edu}{Universitat de Barcelona}} % Your university's name and URL, this is used in the title page and abstract, print it elsewhere with \univname
\department{\href{http://department.university.com}{}} % Your department's name and URL, this is used in the title page and abstract, print it elsewhere with \deptname
\group{\href{http://researchgroup.university.com}{}} % Your research group's name and URL, this is used in the title page, print it elsewhere with \groupname
\faculty{\href{http://mat.ub.edu}{Facultat de Matemàtiques i Informàtica}} % Your faculty's name and URL, this is used in the title page and abstract, print it elsewhere with \facname

\AtBeginDocument{
\hypersetup{pdftitle=\ttitle} % Set the PDF's title to your title
\hypersetup{pdfauthor=\authorname} % Set the PDF's author to your name
\hypersetup{pdfkeywords=\keywordnames} % Set the PDF's keywords to your keywords
}

\begin{document}

\frontmatter % Use roman page numbering style (i, ii, iii, iv...) for the pre-content pages

\pagestyle{plain} % Default to the plain heading style until the thesis style is called for the body content

%----------------------------------------------------------------------------------------
%	TITLE PAGE
%----------------------------------------------------------------------------------------

\begin{titlepage}
\begin{center}

\vspace*{.06\textheight}
{\scshape\LARGE \univname\par}\vspace{1.5cm} % University name
\textsc{\Large Fundamental Principles of Data Science Master's Thesis}\\[0.5cm] % Thesis type

\HRule \\[0.4cm] % Horizontal line
{\huge \bfseries \ttitle\par}\vspace{0.4cm} % Thesis title
\HRule \\[1.5cm] % Horizontal line
 
\begin{minipage}[t]{0.4\textwidth}
\begin{flushleft} \large
\emph{Author:}\\
\href{http://www.johnsmith.com}{\authorname} % Author name - remove the \href bracket to remove the link
\end{flushleft}
\end{minipage}
\begin{minipage}[t]{0.4\textwidth}
\begin{flushright} \large
\emph{Supervisor:} \\
\href{http://www.jamessmith.com}{\supname} % Supervisor name - remove the \href bracket to remove the link  
\end{flushright}
\end{minipage}\\[3cm]
 
\vfill

\large \textit{A thesis submitted in partial fulfillment of the requirements\\ for the degree of MSc in Fundamental Principles of Data Science}\\[0.3cm] % University requirement text
\textit{in the}\\[0.4cm]
\facname\\[2cm] % Research group name and department name
 
\vfill

{\large \today}\\[4cm] % Date
%\includegraphics{Logo} % University/department logo - uncomment to place it
 
\vfill
\end{center}
\end{titlepage}


%----------------------------------------------------------------------------------------
%	ABSTRACT PAGE
%----------------------------------------------------------------------------------------

\begin{abstract}
\addchaptertocentry{\abstractname} % Add the abstract to the table of contents
The current machine learning paradigms consider no further information apart from
the data samples when trying to learn a model that can represent the data and be
used later on to infer information about new samples. During the training process,
the current methods try to minimize the error with respect to the original data, thus
searching the function that best fits the data in an infinite space of functions. However,
the training data has statistical information that can help reduce the searching 
space to a region of it, allowing also to find functions that better fit the data. A recently
proposed data-driven learning paradigm called LUSI tries to take advantage of this information
by preserving some invariants containing interesting statistical properties of the data.
Despite offering good results, the invariants are problem dependent, prior knowledge of the
problem is required when selecting the invariants and it is limited to preserve the information
of only one class at the time. Hence, in this work we propose and study the use of random projections
and random hyperplanes as general use invariants. We observe that the random projections offer
results similar to the original invariants although are a bit limited, whereas the random hyperplanes
fall quite short compared to the other types of invariants. Also, we successfully extend the LUSI
paradigm to multiple classes using the Error Correcting Output Codes framework,
enabling its use in multiclass classification problems.
\end{abstract}

%----------------------------------------------------------------------------------------
%	ACKNOWLEDGEMENTS
%----------------------------------------------------------------------------------------

\begin{acknowledgements}
\addchaptertocentry{\acknowledgementname} % Add the acknowledgements to the table of contents
First and foremost, I would like to thank Dr. Oriol Pujol for giving me the opportunity to work
along with him on this project and supervising my work. Also, I would like to thank him for
constantly encouraging me, giving me new ideas to try when I got stuck on some problem, for
his constant support and for helping me with other academic stuff. It has been a real pleasure
working along with him on this project from the beginning to the end of it.

Second, I would like to thank my parents for unconditionally supporting me and always motivating
me to keep studying the things that I like.

I would also like to thank Pablo Álvarez, who had previously worked on a similar project and whose
work helped me when I got stuck with some issues.

Next, I would like to thank my friends Álvaro, Arseniy and Noah for their constant help, support
and motivation throughout this year. I feel very lucky for having the opportunity to work with them
on different projects throughout the academic year.

Finally but not least, I would like to thank my dear friend Irene for all her help, motivation and
unconditional support, as well as the tips and help that she provided me with to better present the results
of this work and for constantly encouraging me to keep pushing forward while I was working on this thesis.
And my dear friend Otis, for making me take things easier, for his unconditional help and support and for
making my working environment much better in general. And to both of them, thank your for sticking with me while
I was working on this document and for helping me with the experiments.
\end{acknowledgements}



%----------------------------------------------------------------------------------------
%	THESIS CONTENT - CHAPTERS
%----------------------------------------------------------------------------------------

\mainmatter % Begin numeric (1,2,3...) page numbering

\pagestyle{thesis} % Return the page headers back to the "thesis" style

\tableofcontents

% Include the chapters of the thesis as separate files from the Chapters folder
% Uncomment the lines as you write the chapters

% Chapter Template

\chapter{Introduction}

\label{Chapter1}

%----------------------------------------------------------------------------------------
%	SECTION 1: MOTIVATION AND BRIEF DESCRIPTION OF THE PROJECT
%----------------------------------------------------------------------------------------

\section{Motivation and brief description of the project}

The current machine learning paradigms consider no further information apart from the data
samples when trying to learn a model that can represent the data and be used later on
to infer information about new samples. During the training process, the current methods
try to minimize the error with respect to the original data, thus searching the function
that best fits the data in an infinite space of functions. However, the training data has
statistical information that can help reduce the searching space to a region of it, allowing
also to find functions that better fit the data.

A new learning paradigm that takes into account the statistical information of the training data
in the form of statistical invariants has been recently proposed by \cite{Vapnik2019}. Thanks to it,
statistical information of the problem can be used in the learning process, which might
be overlooked by most of the models because some relationships between variables are hard to
spot or require prior knowledge that the model does not have access to. Nonetheless,
it seems that this learning paradigm has not been fully explored or applied that much in practice.

Therefore, in this work we would like to further explore the possible applications of this
paradigm and whether it can be made more general, without requiring prior knowledge of the problem.

%----------------------------------------------------------------------------------------
%	SECTION 2: GOALS AND OBJECTIVES
%----------------------------------------------------------------------------------------

\section{Goals and objectives}

In this thesis we aim to
\begin{enumerate*}[label=(\roman*)]
    \item understand and further explore the application of the invariants in the learning
    problem,
    \item propose new invariants that require no previous knowledge of the problem and
    thus can be applied to multiple domains and
    \item create a small module that implements these invariants and allows them
    to be applied to binary and multiclass classification problems.
\end{enumerate*}

In order to accomplish these goals, we propose a series of milestones that must be achieved
first:

\begin{enumerate}
    \item Understand the original paper and reproduce it, which implies implementing the proposed
    algorithms for learning with statistical invariants and reproducing some of the experiments
    and results. Because there is no source code available, we have to start from scratch.
    \item Propose new invariants and apply them to the same problems as the ones in the paper
    to get an initial idea of how they work.
    \item Build a wrapper around the previously defined binary classifier to enable its use
    in multiclass classification problems.
    \item Experiment with multiclass classification problems to test the proposed invariants and
    compare them to a baseline to see whether they are actually helping or not during the learning
    process.
\end{enumerate}


%----------------------------------------------------------------------------------------
%	SECTION 3: BRIEF SUMMARY OF THE RESULTS
%----------------------------------------------------------------------------------------

\section{Brief summary of the results}

% TODO: Waiting for final results

%----------------------------------------------------------------------------------------
%	SECTION 4: LAYOUT
%----------------------------------------------------------------------------------------

\section{Layout}

This thesis is structured as follows:

\begin{itemize}
    \item Chapter \ref{Chapter1} introduces this work, presenting the main goals that are expected
    to be achieved by the end of it and briefly discussing the obtained results.
    \item Chapter \ref{Chapter2} explains the background work that has inspired this project,
    showing its main contributions and results.
    \item Chapter \ref{Chapter3} proposes a series of invariants that aim to be more general and easy
    to apply to different problems and a method to expand this paradigm to multiclass classification
    problems.
    \item Chapter \ref{Chapter4} studies how the proposed invariants and methods work in practice
    and what results can be achieved with them.
    \item Chapter \ref{Chapter5} briefly discusses what conclusions can be drawn from this work
    and what future work can be done on the topic.
\end{itemize}


% Chapter Template

\newcommand{\Tau}{\mathcal{T}}
\newcommand{\norm}[1]{\lVert #1 \rVert}
\newcommand{\innerprod}[1]{\left< #1 \right>}
\newcommand{\set}[1]{\lbrace #1 \rbrace}

\chapter{Learning using statistical invariants} % Main chapter title
\label{Chapter2}

Given that this work intends to explore the applications of the invariants in the learning
process, we first need to introduce the background work that proposed this new learning
paradigm, which is called LUSI (Learning Using Statistical Invariants).

This chapter intends to provide the necessary background to understand the basis of this work
and an overview of the most relevant aspects of the original paper that presented the LUSI paradigm,
which was proposed by \cite{Vapnik2019}. For further information and more details, please
refer to the original paper.

\section{Weak convergence and the LUSI paradigm}

Supervised machine learning algorithms try to find the best estimate of some conditional probability
function $P(y | x)$, i.e., given a data point $x$, we want to compute the probability that this
point belongs to a particular class $y$.

Classical methods do this by using the strong mode of convergence in the Hilbert space. However,
in the LUSI paradigm this estimation is obtained using the weak mode of convergence. Hence, it is
important to understand the difference between this two modes of convergence and what role
the weak mode of convergence plays in the LUSI paradigm.

\subsection{Strong and weak modes of convergence}

In a Hilbert space, the relationships between two functions $f_1(x)$ and $f_2(x)$ have two
numerical properties:

\begin{enumerate}
    \item The distance between functions
    
    \[
        \rho (f_1, f_2) = \norm{f_1(x) - f_2(x)}
    \]
    
    that is defined by the metric of the $L_2$ space and
    
    \item The inner product between functions
    
    \[
        R(f_1, f_2) = \innerprod{f_1(x), f_2(x)}
    \]
    
    that has to satisfy the corresponding requirements.
\end{enumerate}

These two properties imply two different modes of convergence: a strong one and a weak one. Classical
learning paradigms rely on the strong convergence mode (convergence in metrics), trying to find a
sequence of functions $\set{P_l(y=1 | x)}$\footnote{We focus here in the binary problem setting, i.e. $y\in \{0,1\}$. Thus, $P_l(y=1 | x)$ fully specifies the output probability distribution.} such that

\[
    \lim_{l \to \infty} \norm{P_l(y=1 | x) - P(y=1 | x)} = 0\quad \forall x
\]

The weak mode of convergence (convergence in inner products) is given by

\[
    \lim_{l \to \infty} \innerprod{P_l(y=1 | x) - P(y=1 | x), \psi(x)} = 0\quad \forall \psi(x) \in L_2
\]

Note that this mode of convergence has to take place for \emph{all} functions in the Hilbert space $L_2$.

It is known that the strong mode of convergence implies the weak one, although generally speaking, the
reverse is not true.

\subsection{The LUSI paradigm}

Opposite to the classical learning paradigms, LUSI is based on the weak mode of convergence. It replaces
the infinite set of functions with a set of functions $\mathcal{P} = \set{\psi_1(x), \dots, \psi_m(x)}$
called predicates, which describe some important properties of the desired conditional probability function and
restrict the scope of weak convergence only to the set of functions $\mathcal{P}$. These properties are
called invariants, and can be expressed as the following equalities:

\[
    \int \psi_s P(y=1 | x)dP(x) = \int \psi_s dP(y=1, x) = a_s,\quad s = 1, \dots, m
\]

where $a_s$ is the expected value of the predicate $\psi_s(x)$ with respect to measure
$P(y=1, x)$. These values are unknown but can be estimated using the training data
$\set{(x_i, y_i),\ i = 1, \dots, l}$. Therefore, the previous expression can be rewritten
as follows:

\begin{equation}
    \label{eq:invariant_approximation}
    \frac{1}{l} \sum_{i=1}^l \psi_s(x_i)P_l(y=1 | x_i) \approx a_s \approx \frac{1}{l} \sum_{i=1}^l y_i \psi_s(x_i),\quad
    s = 1, \dots, m
\end{equation}

Simply put, the general idea of the LUSI paradigm is to find an approximation $P_l(y=1|x)$ of the
real conditional probability function in the subset of functions that preserve the invariants
associated to the set of predicates $\mathcal{P}$, reducing effectively the set of candidate functions
to those that satisfy \eqref{eq:invariant_approximation}.

\subsection{Predicate selection}

In order to find this approximation of the conditional probability function, there must exist
some kind of mechanism that allows us to determine which invariants should be used. Luckily,
the authors propose a very simple way to sequentially selecting invariants. Given an approximation
$P_l^m(y=1|x)$ using $m$ invariants and a new predicate $\psi_{m+1}$ which we would to know whether
it should be considered or not. We can compute the following value before adding it:

\begin{equation}
    \label{eq:predicate_selection}
    \Tau = \frac{\left| \sum_{i=1}^l \psi_{m+1}(x_i) P^m_l(y = 1 | x_i) - \sum_{i=1}^l y_i \psi_{m+1}(x_i) \right|}{\sum_{i=1}^l y_i \psi_{m+1}(x_i)}
\end{equation}

If $\Tau \geq \delta$ for some small threshold $\delta$, the new invariant defined by predicate $\psi_{m+1}$
is considered. Otherwise, the expression \eqref{eq:invariant_approximation} is treated as an equality
and the invariant is not considered in the approximation.


\section{Statistical invariants}

A \emph{statistical invariant} is a specific realization of a predicate with statistical meaning.
This means that it captures some sort of statistical information of the data that has to be
conserved when selecting the best approximation of the conditional probability function.

There are different types of statistical invariants, each one of them
providing different information about the data. In this case, the authors have considered
two in particular: the zeroth order and first order moments of the conditional probability
function $P(y=1|x)$. We will briefly discuss each one of them and see what kind of information
they provide.

\subsection{Zeroth order invariant}

Suppose that we are given a binary classification problem, in which the positive instances
are labeled as 1 and the negative ones as 0. The zeroth order invariant would give us information
about the ratio of elements of the positive class. It is defined as follows:

\[
    \psi_0(x) = 1
\]

The logic behind it is the following: the predicate is applied to each single sample in the dataset,
which will yield the vector $(1, \dots, 1) \in \mathbb{R}^l $, where $l$ is the number of samples. Taking into account
expression \eqref{eq:invariant_approximation}, we can see that each element of this vector is multiplied
by the predicted labels (left side) and the true labels (right side). These values are summed and then divided
by $l$, which gives us the proportion of positive predicted elements on the left side and the proportion
of true positive elements on the right side. Notice that the invariant is only taken into account for those
elements whose predicted or true label is positive. Thus, the negative samples are not considered.

\subsection{First order invariant}

Suppose the same case scenario as in the previous subsection. If we apply the first order invariant to
a dataset, we would get the mean or centroid of the positive class. Its mathematical expression is:

\[
    \psi_1(x) = x
\]

Same as before, when this predicate is applied to the dataset it will generate the vector
$(x_1, \dots, x_l) \in \mathbb{R}^l$. Following expression \eqref{eq:invariant_approximation} again,
only the positive true or predicted elements will be considered. Therefore, their values will be summed and
then averaged, yielding indeed the centroid of the positive class (both for the predicted and true labels).

\section{Solving the learning problem}
\label{sect:solvin_learning_problem}

The authors show that in a specific type of Hilbert space called Reproducing Kernel Hilbert Space (RKHS)
the estimate of the conditional probability function can be computed as

\[
    f(x) = A^T \mathcal{K}(x) + c
\]

where $A \in \mathbb{R}^l$ is a vector of coefficients, $\mathcal{K}(x) = (K(x_1, x), \dots, K(x_l, x))^T$
is a vector of functions determined by the kernel associated to the RKHS\footnote{In this case, the authors have
considered the Gaussian Kernel, which is defined as
\[
    K(x, x') = \exp \lbrace -\delta \norm{x - x'}^2 \rbrace,\; \delta > 0
\]
}
and evaluated on the training data, and $c \in \mathbb{R}$ is the bias term.

Additionally, let $Y = (y_1, \dots, y_l)$  be the labels of the training set,  $K \in \mathbb{R}^{l \times l}$
the matrix with elements $K(x_i, x_j),\; i, j = 1, \dots, l$, $\Phi_s = (\psi_s(x_1), \dots, \psi_s(x_l))^T$
the vector obtained from evaluating the $l$ points of the sample using predicate $\psi_s$,
$1_l = (1, \dots, 1) \in \mathbb{R}^l$ a vector of ones and $V \in \mathbb{R}^{l \times l}$ a matrix called
the $V$-matrix, proposed by \cite{Vapnik2015}, which captures some geometric properties of the data\footnote{The
most simple case considers that the $V$-matrix is equivalent to the identity matrix. Also, the authors found
that using the $V$-matrix over the identity matrix didn't improve the results that much, but it was rather the
use of invariants that made the difference. For the sake of consistency, we are going to keep the $V$-matrix in
the expressions as this is how they are supposed to be written, although bear in mind that it can be substituted with
the identity matrix.}.

With all of this information, we can formulate and solve a minimization problem subject to the constraints
\eqref{eq:predicate_selection} which has a closed-form solution. Using its Lagrangian, we can obtain that the
coefficients $A$ are given by:

\[
    A = (A_V - cA_c) - \left( \sum_{s=1}^m \mu_s A_s \right)
\]

where

\begin{equation*}
    \begin{gathered}
        A_V = (VK + \gamma I)^{-1} VY \\
        A_c = (VK + \gamma I)^{-1} V1_l \\
        A_s = (VK + \gamma I)^{-1} \Phi_s,\quad s = 1, \dots, n 
    \end{gathered}
\end{equation*}

In this case, $\gamma$ controls the amount of regularization applied so that the resulting matrix is not singular.

The values of $c$ and the $m$ coefficients $\mu_s$ can be obtained solving the following system of equations:

\begin{equation*}
    \begin{gathered}
        c [1_l^T VKA_c - 1_l^T V 1_l] + \sum_{s=1}^m \mu_s [1_l^T VKA_s - 1_l^T \Phi_s] = [1_l^T VKA_V - 1_l^T V Y] \\
        c [A_c^TK\Phi_k - 1_l^T\Phi_k] + \sum_{s=1}^m \mu_s A_s^T K \Phi_k = [A_V^T K \Phi_k - Y^T \Phi_k],\quad k=1, \dots, m
    \end{gathered}
\end{equation*}

This algorithm, which uses $m$ invariants, is called vSVM\&$\text{I}_m$ if the $V$-matrix is used
or SVM\&$\text{I}_m$ in case it is not.

\section{Overview of the LUSI algorithm}

Following the mathematical formulations from the previous sections, let us now present an overview of
the LUSI algorithm.

Consider the following learning method given a training sample $(x_i, y_i),\; i=1, \dots, l$ and
$\psi_k(x),\; k=1, \dots, m$ predicates:

\begin{enumerate}[label=\textbf{Step \arabic*:}]
    \item Construct SVM or vSVM estimate of conditional probability as described in section
    \ref{sect:solvin_learning_problem}, without considering the predicates.
    \item Find the maximal disagreement value $\Tau_s$ as defined in \eqref{eq:predicate_selection}
    for vectors
    \[
        \Phi_k = (\psi_k(x_1), \dots, \psi_k(x_l))^T,\quad k=1, \dots, m
    \]
    \item If $\Tau_s > \delta$, add the invariant associated to the predicate $\psi_s$; otherwise stop.
    \item Find a new approximation of the conditional probability function and go back to \textbf{Step 2};
    otherwise stop.
\end{enumerate}

\section{Main results and limitations}

According to the original work, LUSI yields quiet good results overall, reducing the error and thus improving
the accuracy of the models that use it compared to models that do not use statistical invariants. Moreover,
the authors state that it can reduce the number of necessary training examples to obtain a good approximation
of the conditional probability function. Therefore, this method could be very useful in cases in which the
amount of available training data is small.

However, this new learning paradigm presents some important flaws:

\begin{enumerate}
    \item The selected invariants are problem dependent, which means that it is hard to have general invariants
    that can both be applied to multiple problems without requiring any kind of previous knowledge
    and yield good results.
    \item Often, the invariant selection is a ``black-art'' due to the fact that they can either be
    very esoteric or require a lot of knowledge about the specific problem which might be hard
    to obtain. Hence, some craftsmanship is required when selecting the invariants that are going
    to be used.
    \item Considering expressions \eqref{eq:invariant_approximation} and \eqref{eq:predicate_selection}, we
    can clearly see that the invariants can only consider the statistical information of
    the positive class. The way that the values are computed make it hinder the application of the invariants
    to the negative class and to multiclass classification problems since there is no positive class in this kind
    of scenarios. This a very serious drawback of this method that needs to be further addressed.
\end{enumerate}

Thus, even though the use of invariants can improve the obtained results, they also introduce some additional
complexity to the task because they require extra knowledge that may not be accessible.
 
% Chapter Template

\newcommand{\reals}[1]{\mathbb{R}^{#1}}

\chapter{Proposals} % Main chapter title
\label{Chapter3}

As we have already seen in Chapter \ref{Chapter2}, LUSI introduces a new data-driven learning
paradigm which aims to find better approximations of the conditional probability function using
statistical invariants. However, these invariants are often problem dependent and choosing the
appropriate ones is not an easy task, as there are many possible ones and some of them are not
straightforward to come up with.

Thus, our main goal with this work is to create a series of new invariants which we expect to be of
general use, as well as automatizing the selection process of the invariants that should be considered
for a given problem and extending the LUSI paradigm to multiclass problems\footnote{Even though in the
original paper the authors state that the method can be applied to multiclass problems, they do not go
into too much detail of how this is done.}.

In this chapter we are going to present our proposals, which are a two new invariants based on randomness
which aim to be more general than the original ones and an extension of the LUSI paradigm to multiclass problems
using Error Correcting Output Codes (ECOC).

\section{Random invariants}

Our first proposal is a series of random invariants, which are invariants that have some sort of random
process inside of them but that aim to preserve some sort of statistical information of the data. This way,
we aim to greatly reduce the amount of necessary prior knowledge of the problem when choosing which invariants
to use, treating it just like any other hyperparameter of a machine learning model. In this work, we propose
an invariant based on random projections as well as another one based on random hyperplanes.

\subsection{Random projections}

Random projections have been frequently used in the machine learning field to perform dimensionality reduction
in a faster and computationally less expensive way than other techniques (i.e., PCA) as studied in 
\cite{Dasgupta2000} and \cite{BinghamManila2001}.

In our case, the random projection invariant offers a glimpse of the data from a different viewpoint
and compresses the data into a single dimension as if it was viewed from that particular point.
Intuitively, this can be observed in figure \ref{fig:random_projections_example}.

More formally speaking, consider a data point $x \in \reals{d}$ and a projection vector
$p \sim \mathcal{N}(\mu, \Sigma)$, where the multivariate normal distribution has mean $\mu \in \reals{d}$
and covariance matrix $\Sigma \in \reals{d \times d}$. We can define the random projection invariant as follows:

\begin{equation}
    \label{eq:random_projection_invariant}
    \psi_{r.p.}(x) = x p
\end{equation}

As we can see in expression \eqref{eq:random_projection_invariant}, we are computing the dot product between
the data point and the projection vector. When this invariant is used in expression \ref{eq:invariant_approximation}
it will try to preserve the centroid of the positive class in the new projected space. Hence, it is a variation
of the first order invariant. When using multiple random projections as invariants, we expect that the centroid
of the positive class is preserved across different views of the data.

\begin{figure}[h]
    \centering
    \includegraphics[width=\textwidth]{thesis/Figures/random_projections_example}
    \caption{Example of a dataset and two random projections of it. In the first projection, we can see
    that the points from the red and blue classes are almost totally overlapped, whereas in the second projection
    they are completely separable.}
    \label{fig:random_projections_example}
\end{figure}

\subsection{Random hyperplanes}

The second invariant that we propose is the random hyperplane. With it, we aim to create two partitions of
the original data: the samples that are on the right side of the hyperplane and the ones on the left, which
we will consider as the positive and the negative samples, respectively. Opposite to the previous invariant, which
produced a real value when applied to a point $x$, this one produces a discrete value $\set{0, 1}$ based on the
relative position of the point with respect to the normal vector of the hyperplane.
Figure \ref{fig:random_hyperplanes_example} shows an example of how this invariant works when applied to
an example dataset.

Formalizing the previous explanation, consider an arbitrary point from the sample $x_c$. Let
$n \sim \mathcal{N}(\mu, \Sigma)$ be the normal vector of the hyperplane that contains $x_c$, where the
multivariate normal distribution has mean $\mu \in \reals{d}$ and covariance matrix $\Sigma \in \reals{d \times d}$.
We can define the random hyperplane invariant as

\begin{equation}
    \label{eq:random_hyperplane_invariant}
    \psi_{r.h.}(x) =
    \begin{cases}
        1 & \text{if $(x - x_c)n \geq 0$ }\\
        0 & \text{otherwise}
    \end{cases}
\end{equation}

Considering the previous expression, we can clearly see that this invariant will yield a vector of zeroes
and ones when applied to the data sample. If we also take into account expression \eqref{eq:invariant_approximation},
we can deduce that this invariant will try to preserve the proportion of positive samples that fall on the right
side of the hyperplane. Consequently, we can see that this is a variation of the zeroth order invariant.
The main difference is that we are now trying to preserve the proportion of positive elements in a subspace
of the original space (the subspace formed by the samples that are on the same side as the normal vector of the
hyperplane), instead of trying to preserve it in the whole space.

\begin{figure}[h]
    \centering
    \includegraphics[width=\textwidth]{thesis/Figures/random_hyperplanes_example.png}
    \caption{Example of the hyperplanes invariant. In the left image, we can see the original data. In
    the middle and right images we can see two hyperplanes that divide the data in elements which are
    on the right and the left of the hyperplane, labeled as 1 and 0 respectively. Note that these
    new labels do not necessarily match the original ones.}
    \label{fig:random_hyperplanes_example}
\end{figure}

\section{Extending the LUSI paradigm to multiclass problems}


\chapter{Experimentation and results}
\label{Chapter4}

After having explained the necessary theoretical background and presented the proposals
of this work, it is time that we put them into action to see how they perform. First, we are going
to test our proposals on some toy datasets in order to better understand them. After that, we are going
to try the proposed invariants as well as the multiclass extension with ECOC on real data to see how
well they perform when comparing them to the original invariants proposed in \cite{Vapnik2019}.

\section{Experimenting with toy problems}

In this section we will perform a set of simple experiments with some toy datasets. In this case, we have
considered the circles and moons datasets, which can be seen in figure \ref{fig:toy_datasets}. Both of
these problems are available in \texttt{scikit-learn}, which means that we will be able to easily generate
our custom datasets with the available functions. When experimenting with these problems we aim to:

\begin{itemize}
    \item Compare the original invariants with the ones that have been proposed in this work.
    \item Compare the original version of the LUSI algorithm with the ECOC version to see if there is
    any significant difference in a binary setting.
    \item Discover whether some types of invariants are more likely to be selected when considering all types
    of invariants.
\end{itemize}

\begin{figure}[H]
    \centering
    \includegraphics[width=\textwidth]{thesis/Figures/toy_datasets.png}
    \caption{Toy datasets used in the experimentation.}
    \label{fig:toy_datasets}
\end{figure}

For this experimentation, we have generated one dataset for each problem with a fixed seed. The experimental
parameter settings for each type of problem can be seen in table \ref{tab:toy_problems_experiments}. In an experiment,
we split the data in training and test, keeping 10\% of the data in the training partition and the reamining
90\% in the test partition. We have performed these experiments using Vapnik's invariants and the ones
that we have proposed. As for Vapnik's invariants, we have considered the zeroth and first order invariants, which are

\[
    \psi_0(x) = 1,\quad \psi_1(x) = x_1,\quad \psi_2(x) = x_2
\]

\subsection{Comparing the different invariant types and versions of LUSI}

In order to compare the two versions of the LUSI algorithm and the different invariants
we have trained a model for each version of LUSI and each invariant type on both problems with a fixed
initial random state. We have visualized the decision boundaries in each case in order to see if there is any
significant difference between them. Also, for the LUSI version we have reported the mean number of selected
invariants of each type by repeating the experiment with 10 different initial states. We have limited the
maximum number of invariants of each model to 3 because even though we can generate infinite new random
invariants, we cannot do the same with the original invariants.

\begin{table}[h]
\centering
\begin{tabular}{llr}
\textbf{Problem} & \textbf{Parameter}  & \textbf{Value} \\ \hline
Circles          & \texttt{n\_samples} & 1000           \\
                 & \texttt{noise}      & 0.1            \\
                 & \texttt{factor}     & 0.5            \\
Moons            & \texttt{n\_samples} & 1000           \\
                 & \texttt{noise}      & 0.05          
\end{tabular}
\caption{Parameter settings of the toy problems.}
\label{tab:toy_problems_experiments}
\end{table}

Figures \ref{fig:circles_decision_boundary} and \ref{fig:moons_decision_boundary} show the decision
boundaries generated by the original version of the LUSI algorithm whereas figures
\ref{fig:circles_decision_boundary_ecoc} and \ref{fig:moons_decision_boundary_ecoc} display the decision
boundaries of the ECOC version. For the circles problem, we can see that all invariants produce
very similar decision boundaries. In the case of ECOC version, this decision boundary seems to be much
closer to the points of the blue class than in the case of the original algorithm, where the decision boundary
is a bit wider. As for the moons problem, we can observe that the decision boundary is not perfect
in any of the versions of the algorithm since some points fall into the region of the opposite class,
probably caused by to the geometric shape of the both classes. For this problem, the decision boundary seems
to be more accurate using the original algorithm. This is especially true in the case of the random hyperplanes,
where the decision boundary is better adjusted in the case of the original algorithm, whereas it seems that it
has been overfitted when using the ECOC algorithm because we can observe some ``decision islands'' for the blue
class. Overall, it seems that Vapnik's invariants and the random projections produce very similar results
regardless of which version of the algorithm is applied. Hence, if we use any of these two invariants, we
could apply any version of the algorithm and get similar results in a binary classification problem. In the
case of the random hyperplanes, there would be some difference between the results obtained with each version
of the algorithm. As we have seen, depending on the problem, we could get similar results to the ones obtained
using the other two types of invariants.

The idea that Vapnik's invariants and the random projections are similar can be further explored. Figure
\ref{fig:toys_small_num_selected_invariants} show how many invariants have been selected for each problem
on average. We can observe that the mean number of selected invariants is the same for Vapnik's invariants
and the random projections, whereas the number of selected invariants is equal to the maximum number of
invariants in the case of the random hyperplanes. Thus, it seems that the number of invariants that can be
chosen when using Vapnik's invariants and the random projections is limited by the number of dimensions of the
data, as selecting more does not provide any new information. In the case of the random hyperplanes, this is
generally not true as we could keep adding more invariants of this type. This might be caused by the
fact that there is a very large number of hyperplanes that separate the data in two partitions and that
can be used to preserve the proportion of elements that fall on the right side of the hyperplane.
Because of this, many invariants of this type can be selected.

\subsection{Exploring the bias towards certain types of invariants}

Now, we would like to study the scenario in which all types of invariants are considered to see if there is
any kind of bias towards particular types of invariants. For this purpose, we have run a similar experiment to
the previous one using the original version of the LUSI algorithm. Using both problems, we have fit 10 models
using different random states and setting the maximum number of invariants to 50. In each experiment, the model could
choose among all of the invariant types. For each run, we have computed how many invariants of each
type were selected.

A summary of the results can be seen in figure \ref{fig:toys_mean_num_selected}, where the number of
selected invariants has been averaged for each problem. We can see that the models have not selected any of
Vapnik's invariants. On average, they have chosen 2 random projections per problem, which once again matches
the number of dimensions that these problems have. The models have selected 48 random hyperplanes on average
per problem, which gives more strength to the hypothesis that the number of hyperplane invariants that
can be selected is very large, potentially infinite. Because of this, we can see that when considering
all types of invariants at the same time, it is more likely that the hyperplanes invariant will be selected
because it can contribute with more information.

\begin{figure}[H]
    \centering
    \includegraphics[width=\textwidth]{thesis/Figures/circles_decision_boundaries.png}
    \caption{Decision boundaries in the circles problem using the original LUSI algorithm with each type of
    invariant on the training and test sets.}
    \label{fig:circles_decision_boundary}
\end{figure}

\begin{figure}[H]
    \centering
    \includegraphics[width=\textwidth]{thesis/Figures/moons_decision_boundaries.png}
    \caption{Decision boundaries in the moons problem using the original LUSI algorithm with each type of
    invariant on the training and test sets.}
    \label{fig:moons_decision_boundary}
\end{figure}

\begin{figure}[H]
    \centering
    \includegraphics[width=\textwidth]{thesis/Figures/circles_decision_boundaries_ecoc.png}
    \caption{Decision boundaries in the circles problem using the ECOC version of the LUSI algorithm with each type
    of invariant on the training and test sets.}
    \label{fig:circles_decision_boundary_ecoc}
\end{figure}

\begin{figure}[h]
    \centering
    \includegraphics[width=\textwidth]{thesis/Figures/moons_decision_boundaries_ecoc.png}
    \caption{Decision boundaries in the moons problem using the ECOC version of the LUSI algorithm with each type
    of invariant on the training and test sets.}
    \label{fig:moons_decision_boundary_ecoc}
\end{figure}

\begin{figure}[H]
    \centering
    \includegraphics[width=0.8\textwidth]{thesis/Figures/num_selected_invariants.png}
    \caption{Mean number of selected invariants per problem. The maximum number of invariants was set to 3.}
    \label{fig:toys_small_num_selected_invariants}
\end{figure}

\begin{figure}[H]
    \centering
    \includegraphics[width=\textwidth]{thesis/Figures/mean_num_selected.png}
    \caption{Mean number of selected invariants when the maximum number of invariants is 50 and the three types of
    invariants are considered simultaneously.}
    \label{fig:toys_mean_num_selected}
\end{figure}

\section{Experimentation on real data}

Once we have got a glimpse of how the proposals compare to the original work, we would like to test them
on real problems to see how they perform compared to the base version.

\subsection{Experimental settings}
{\bf Data:} Datos y tabla resumen de los datasets\\
{\bf Baselines:} Que métodos vas a comparar.
SVM sin invariantes
SVM con invariantes Vapnik
SVM con invariantes Vladis

Todos con el mismo sistema de selección.
\\
{\bf Methodology and performance metrics:} XValidation? cuanto, como mides los resultados?

Accuracy promedio i desviación estandar.
\\
{\bf Experimental and parameter setting:}
Haces parameter tunning?

Que experimentos propones?

1- Experimento base: Todos los datos con los modelos a comparar\\
2- Experimento con reduccion de datos.

\begin{table}[H]
\centering
\begin{tabular}{lrrr}
\textbf{Problem} & \textbf{Num. examples} & \textbf{Attributes} & \textbf{Classes} \\ \hline
Balance scale    & 625                    & 4                   & 3                \\
Ecoli            & 336                    & 8                   & 8                \\
Glass            & 214                    & 9                   & 7(6)\tablefootnote{According to the dataset's documentation, there are 7 classes. However, there is one that does not have any elements. Therefore, the actual number of classes for this problem is 6.}                \\
Iris             & 150                    & 4                   & 3                \\
Yeast            & 1484                   & 8                   & 10              
\end{tabular}
\caption{Caption}
\label{tab:problems_description}
\end{table}


\subsection{Results}

% Please add the following required packages to your document preamble:
% \usepackage{graphicx}
\begin{table}[h]
\centering
\resizebox{\textwidth}{!} &
  10\% &
  \multicolumn{1}{c|}{100\%} &
  10\% &
  \multicolumn{1}{c|}{100\%} &
  10\% &
  \multicolumn{1}{c|}{100\%} &
  10\% \\ \hline
\multicolumn{1}{|c|}{Balance scale} &
  \multicolumn{1}{c|}{$91.47 \pm 0.38\%$} &
  $88.80 \pm 1.73\%$ &
  \multicolumn{1}{c|}{$91.47 \pm 0.38\%$} &
  $87.73 \pm 2.10\%$ &
  \multicolumn{1}{c|}{$91.73 \pm 0.38\%$} &
  $83.73 \pm 7.17\%$ &
  \multicolumn{1}{c|}{$90.13 \pm 1.51\%$} &
  $76.00 \pm 12.77\%$ \\ \hline
\multicolumn{1}{|c|}{Ecoli} &
  \multicolumn{1}{c|}{$85.29 \pm 2.08\%$} &
  $71.57 \pm 1.39\%$ &
  \multicolumn{1}{c|}{$85.29 \pm 1.20\%$} &
  $72.55 \pm 4.22\%$ &
  \multicolumn{1}{c|}{$84.15 \pm 2.05\%$} &
  $67.32 \pm 4.03\%$ &
  \multicolumn{1}{c|}{$75.98 \pm 6.28\%$} &
  $63.89 \pm 9.61\%$ \\ \hline
\multicolumn{1}{|c|}{Glass} &
  \multicolumn{1}{c|}{$72.87 \pm 7.91\%$} &
  $56.59 \pm 4.78\%$ &
  \multicolumn{1}{c|}{$72.87 \pm 7.91\%$} &
  $45.74 \pm 6.67\%$ &
  \multicolumn{1}{c|}{$72.09 \pm 7.44\%$} &
  $52.45 \pm 7.03\%$ &
  \multicolumn{1}{c|}{$71.06 \pm 7.60\%$} &
  $45.48 \pm 9.50\%$ \\ \hline
\multicolumn{1}{|c|}{Iris} &
  \multicolumn{1}{c|}{$96.67 \pm 0.00\%$} &
  $93.33 \pm 2.72\%$ &
  \multicolumn{1}{c|}{$95.56 \pm 1.57\%$} &
  $90.00 \pm 4.71\%$ &
  \multicolumn{1}{c|}{$95.19 \pm 1.66\%$} &
  $77.78 \pm 22.93\%$ &
  \multicolumn{1}{c|}{$88.52 \pm 11.56\%$} &
  $84.81 \pm 9.04\%$ \\ \hline
\multicolumn{1}{|c|}{Yeast} &
  \multicolumn{1}{c|}{$53.76 \pm 1.04\%$} &
  $47.92 \pm 2.08\%$ &
  \multicolumn{1}{c|}{$53.87 \pm 1.20\%$} &
  $47.92 \pm 2.14\%$ &
  \multicolumn{1}{c|}{$52.53 \pm 5.44\%$} &
  $45.68 \pm 4.60\%$ &
  \multicolumn{1}{c|}{$50.39 \pm 5.33\%$} &
  $37.82 \pm 7.92\%$ \\ \hline
\end{tabular}%
}
\caption{Caption}
\label{tab:results_accuracies_errors}
\end{table}

\begin{figure}
    \centering
    \includegraphics[width=0.7\textwidth]{thesis/Figures/invariants_performance.png}
    \caption{Caption}
    \label{fig:invariants_performance}
\end{figure} 
% Chapter Template

\chapter{Conclusions} % Main chapter title
\label{Chapter5}

Summing up, in this work we have studied the application of statistical invariants to the learning
problem through a new learning paradigm called LUSI. We have seen the theoretical background
that it is based on as well as the potential benefits that it can bring and its limitations.
We have seen that the invariants are problem dependent and choosing the most appropriate ones
is a hard task which requires prior knowledge. Also, the original version of the LUSI algorithm
can only consider one class at the time, which hinders its application to more complex problems.

With these limitations in mind, we have proposed two new invariants, which are random projections
and random hyperplanes. These invariants use random processes in order to keep statistical information
of the data, are of general use and require no previous knowledge of the problem. Also, we have
proposed an extension of the original LUSI algorithm using ECOC so that it can be applied to any
class in both binary and multiclass classification problems.

We have seen that the extended version of LUSI using ECOC has been successfully applied to both
binary and multiclass classification problems without negatively impacting the final results.

However, after performing some experiments with the new invariants we have seen that the obtained
results were not as good as expected. The random projections offered similar results to the original
invariants, although they were slightly worse in general. As far as the random projections go, the
results that we obtained using them were discouraging, as it was the type of invariant that performed
the worst in almost all scenarios.

Also, even though it was one of the main goals of this thesis, we were not able to automatize
the invariants selection process, as we still have to manually choose which invariants need to be
applied to a given problem. Thus, it still remains an open topic.

In conclusion, we believe that this new version of LUSI using ECOC is an important step in making
this data-driven paradigm more accessible and easy to apply for the machine learning community.
Also, the random projections are an interesting proposal that can be applied to multiple problems
and achieve overall good results, although there are other invariants that might work better.
Nonetheless, it is an important example of how new invariants can be created.

\section{Future work}

Even though we have accomplished some of the goals that we had set at the beginning of this thesis,
there is still a lot of room for improvement. Moreover, there are some other lines of work we wanted
to explore but we couldn't because of the time limitations of this project or because they ended
up being out of scope. Thus, here are some proposals of future work that we believe are of great
interest:

\begin{itemize}
    \item The current formulation of the LUSI algorithm as a system of equations is quite cumbersome.
    We could rewrite this formulation so that an iterative algorithm like SGD can be applied to it.
    This way, we could enable the application of this data-driven paradigm to different types of
    machine learning algorithms. We believe that this task is quite difficult and we would need
    to explore the possible limitations of this new formulation, or whether it can be done at all.
    \item The original work only proposed invariants up to the first order. It would be interesting
    to see whether it makes sense to apply higher order invariants to the learning problem and if
    they can be used to achieve better results.
    \item Related to the previous points, coming up with invariants that can be applied to text
    or images is a task that would further enable LUSI to be applied to new domain problems. However,
    it would first need to be reformulated.
    \item We believe that the random projections invariant can be improved by constructing an orthogonal
    space from random vectors and projecting the data in it. This could be done using an orthonormalization
    process like the Gram-Schmidt process.
\end{itemize}


%----------------------------------------------------------------------------------------
%	THESIS CONTENT - APPENDICES
%----------------------------------------------------------------------------------------

% \appendix % Cue to tell LaTeX that the following "chapters" are Appendices

% Include the appendices of the thesis as separate files from the Appendices folder
% Uncomment the lines as you write the Appendices

% % Appendix A

\chapter{Frequently Asked Questions} % Main appendix title

\label{AppendixA} % For referencing this appendix elsewhere, use \ref{AppendixA}

\section{How do I change the colors of links?}

The color of links can be changed to your liking using:

{\small\verb!\hypersetup{urlcolor=red}!}, or

{\small\verb!\hypersetup{citecolor=green}!}, or

{\small\verb!\hypersetup{allcolor=blue}!}.

\noindent If you want to completely hide the links, you can use:

{\small\verb!\hypersetup{allcolors=.}!}, or even better: 

{\small\verb!\hypersetup{hidelinks}!}.

\noindent If you want to have obvious links in the PDF but not the printed text, use:

{\small\verb!\hypersetup{colorlinks=false}!}.

\section{Experimenting with toy problems}

Comparar invariantes de Vapnik con los nuevos

Datasets

- Moons

- Cluster points

- Circles


Cosas que se podrían probar: Influencia de los hiperparámetros


\section{Experimentation on real data}
\subsection{Experimental settings}
{\bf Data:} Datos y tabla resumen de los datasets\\
{\bf Baselines:} Que métodos vas a comparar.
SVM sin invariantes
SVM con invariantes Vapnik
SVM con invariantes Vladis

Todos con el mismo sistema de selección.
\\
{\bf Methodology and performance metrics:} XValidation? cuanto, como mides los resultados?

Accuracy promedio i desviación estandar.
\\
{\bf Experimental and parameter setting:}
Haces parameter tunning?

Que experimentos propones?

1- Experimento base: Todos los datos con los modelos a comparar\\
2- Experimento con reduccion de datos.


\subsection{Results}

%\include{Appendices/AppendixB}
%\include{Appendices/AppendixC}

%----------------------------------------------------------------------------------------
%	BIBLIOGRAPHY
%----------------------------------------------------------------------------------------

\printbibliography[heading=bibintoc]

%----------------------------------------------------------------------------------------

\end{document}  
